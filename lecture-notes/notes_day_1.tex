\documentclass[fleqn]{article}
\setlength{\parskip}{\baselineskip}%
\setlength{\parindent}{0pt}%

\usepackage[table]{xcolor}
\usepackage{siunitx}
\usepackage{tabularx}
\usepackage{float}
\usepackage{caption}
\usepackage{graphicx}
\usepackage{amsmath}

\begin{document}
\setlength{\mathindent}{0pt}
\section*{Number Systems \& Base Conversions}
hexadecimal shortens binary

Three number systems:
\begin{itemize}
    \item Binary
    \item Decimal (10)
    \item Hexadecimal (base 16)
\end{itemize}

\section*{Decimal}
numbers are sum of digits, each multiplied by a power of 10
\[735 = 700 + 30 + 5\]
\[700 = 7 \times 10^{2}\]
Index 2, 1, 0

\section*{Hexadecimal}
16 digits: 0-9, A-F

In base 10, A = 1010 (10), B = 1011 (11), ..., F=1111 (15)

Split into two sections, each with powers of 2

Each digit contains 4 bits

$ 15 _{10}  $ means 15 in decimal

\section*{Convert Decimal to Binary}
See if power numbers (64, 32, 16, 8) fit in 35?

32 $\leq$ 35, so 1 under 32, then $35-32=3$

\section*{Hex to Binary}
Split into four digit segments

1111=0xf

\section*{Binary to Hex}
Conversion same except split into separate charts of 4 digits

If over ten but $<$ 16, convert to letter

Concat, dont sum separate charts

11011001=0xD9

0x prefix represents a hexadecimal number

Also, 0xD9=D9h, h suffix also means hexadecimal

\section*{Hex to Binary}
Split each digit into chart of 4 binary digits

Only need 4 because hex only goes to 15 anyways

Also just concat at the end
\end{document}
