\documentclass[fleqn]{article}
\setlength{\parskip}{\baselineskip}%
\setlength{\parindent}{0pt}%

\usepackage[table]{xcolor}
\usepackage{siunitx}
\usepackage{tabularx}
\usepackage{float}
\usepackage{caption}
\usepackage{graphicx}
\usepackage{amsmath}

\begin{document}
\setlength{\mathindent}{0pt}
\section*{What is Computer Architecture/Org}
\subsection*{Arch}
Design, structure, functionality of computer usually at high level

What a computer does

Interaction between components in system and hard/software
\subsection*{Org}
How works together

Lower level

Circuit design, signal pathways, processor design, control logic

\subsection*{Processor types}
x86 for Intel AMD, MIPS workstations/servers, ARM mobile

Windows 64 bit could be a superset of 32 bit; maybe on linux

\subsection*{x86 cpu arch}
Data flows between ALU, I/O, CPU, memory

ALU: Arithmetic Logic Unit (Combinational Logic)


Driver: tells how to talk to device

\subsection*{Registers}
Register: a unit of memory inside the CPU itself; smallest data holding unit in processor

Registers more efficient than RAM

Move data into register to do operations on data

EAX = accumulator

EBX = base

ECX = counter

EDX = data

ESI = source index

EDI = destination index

\subsection*{nasm syntax}
Data usually goes right to left
\end{document}
